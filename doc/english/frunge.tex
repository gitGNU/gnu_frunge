 \documentclass[
   titlepage=no]{scrartcl} %cuſtom titlepage

 \usepackage{
   polyglossia, %hyphenation
   relsize, %\larger and \smaller text ſize
   verbatim, %quotes
   xcolor, %Driver-independent color extenſions
   xspace, %guesses space after commands
   xltxtra} %xetex enhancements and fixes

 \usepackage[
   breaklinks,
   colorlinks=true,
   pdfborder=0 0 0,
   pdftitle={Frunge Font Documentation},
   linkcolor=blue,
   urlcolor=blue,
   xetex]{hyperref}

%% font settings
 \setmainfont{Linux Libertine} %without O (this is a feature/dirty fix)
 \setsansfont[Scale=MatchLowercase]{Linux Biolinum O}
 
%% polyglossia settings
 \setmainlanguage{english}
 \frenchspacing
 \setotherlanguage[spelling=new, latesthyphen=true]{german}
 % pleaſe add babelshorthands=true if TeXLive 2009 is realeaſed

%% text commands
 \newcommand*\team{Dennis Heidſiek\and Chriſtian Kluge\and Martin Roppelt\and Arno 
Trautmann}

 \newcommand*\smiley{{\larger[3]☺}}
 \newcommand*\frowney{{\larger[3]☹}}
 \newcommand*\winkey{;)}

 \newcommand*\ipa[1]{\mbox{#1}} %another font would be nice
 \newcommand*\pkg[1]{\textsf{#1}}

  \newcommand\frungepangram{\textgerman{\IfFileExists{frunge-pangram.tex}{Im finſteren Jagdſchloſs am offenen Felsquellwaſſer patzte der affig-flatterhafte kauzig-höf\/liche Bäcker über ſeinem verſifften kniffligen C-Xylophon-Fuß und ſchrie: „\textsc{Auweia! Mein Fuß!}“}{Pangramm-Platzhalter}}}


 %\usepackage{syntonly}
 %\syntaxonly

 %\includeonly{}


 \subject{The Documentation of} \title{The Frunge Font Family}
 \author{\team}

 \addtokomafont{section}{\LARGE}




 \begin{document}

 \begin{comment}
 Engliſh long s rules:
 End of word: s / Elſe: ſ
 with f: sf/fs
 with daſh at line end: ſ-
 with apostrophe: s’
 Italic writing: ſſ -> ſs
 17th century: ſb ſk -> sb sk
 \end{comment}


 \maketitle
 \thispagestyle{empty}
 \begin{abstract}
 \large Copyright 2009 Dennis Heidſiek 
(\href{mailto:heidsiekb@aol.com}{heidsiekb@aol.com}), Martin Roppelt 
\break (\href{mailto:m.p.roppelt@web.de}{m.p.roppelt@web.de}) and Arno 
Trautmann (\href{mailto:arno.trautmann@gmx.de}{arno.trautmann@gmx.de})

 Permiſſion is granted to copy, diſtribute and\,/\,or modify this 
document under the terms of the GNU Free Documentation Licenſe, Version 
1.3 or any later verſion publiſhed by the Free Software Foundation; with 
no Invariant Sections, no Front-Cover Texts, and no Back-Cover Texts. A 
copy of the licenſe is included in the ſection entitled 
“\hyperlink{fdl}{GNU Free Documentation Licenſe}”.

This work is licenſed under the Creative Commons Attribution-ShareAlike 
3.0 Unported Licenſe. To view a copy of this licenſe, viſit 
\href{http://creativecommons.org/licenses/by-sa/3.0/legalcode}{this 
page}; or, (b) ſend a letter to Creative Commons, 171 
2\textsuperscript{nd} Street, Suite 300, San Franciſco, California, 
94105, USA.

%If we plan to include (non-trivial?) ſource code:
 \begin{comment}
 This document is free ſoftware: you can rediſtribute it and\,/\,or 
modify it under the terms of the GNU General Public Licenſe as publiſhed 
by the Free Software Foundation, either verſion 3 of the Licenſe, or (at 
your option) any later verſion.

This document is diſtributed in the hope that it will be uſeful, but 
WITHOUT ANY WARRANTY; without even the implied warranty of 
MERCHANTABILITY or FITNESS FOR A PARTICULAR PURPOSE. See the GNU General 
Public Licenſe for more details.

You ſhould have received a copy of the GNU General Public Licenſe along 
with this document. If not, see: http://www.gnu.org/licenses/gpl.html
 \end{comment}

 \end{abstract}
 \clearpage
 \tableofcontents
 \clearpage

 \addsec{Frunge Fonts Project}
 We at the Frunge Fonts Project want to create free, high-quality and 
Unicode complete ſubſtitutes to the proprietary blackletter fonts. 
Frunge is an abbreviation of the German “FRei UNd GEbrochen” (= free and 
blackletter).

Therefore it is pronounced \ipa{[fʁʊŋə]}. But as we tend to pronounce 
the Gentium SIL not as \ipa{[dʒɛntɪʊm]}, but rather as \ipa{[gɛntɪʊm]} 
or even \ipa{[gɛntsɪʊm]}, you may feel free to pronounce it as 
\ipa{[frʌndʒ]}, or as you like it.

We alſo found out, that ſome online dictionaries define frunge as “neck 
maſſage”. This is what it gives us when we finiſhed a font \smiley


\addsec{We Care About Your Freedom}
 Blackletter fonts are uſually very expenſive. Even if not, you may not 
ſtudy, change or even rediſtribute modified verſions \frowney

Becauſe we didn’t want to wait until thoſe publiſhers relicenſe their 
fonts as free ſoftware, we ſtarted this project. A little bit preſſure 
can help them, as well \winkey

Our fonts are free ſoftware, i.\,e. you may
 \begin{itemize}
  \item uſe
  \item ſtudy
  \item modify\,/\,improve and
  \item rediſtribute them.
 \end{itemize}

Uſually you can get them without fee.

We want to enſure that our fonts cannot be made proprietary again, ſo we 
choſe to licenſe them under a 
\href{http://en.wikipedia.org/wiki/Copyleft}{copyleft} licenſe:

To avoid licenſe proliferation and to enſure greateſt font combining 
poſſibilities, our fonts are licenſed under GPL, OFL and CC-BY-SA?.
 %If you feel that this is not enough, pleaſe 
\hyperlink{contact}{contact} us.

If you want to rediſtribute our fonts (probably changed), have a look at 
the full licenſe text(s) to make ſure, that you meet the conditions!

To maximize your freedom, we added exceptions to our licenſes to avoid 
ſome reſtrictions of the vanilla licenſes:

 GPL:
 \begin{itemize}
  \item you may embed this font

 OFL:
  \item you may uſe our font name and
  \item you may ſell our fonts unbundled with other ſoftware
 \end{itemize}


 \addsec{Feature Goals}
 \begin{itemize}
  \item \XeTeX\ Math Fonts
  %\item Language specific typography
  %\item r rotunda
  \item vowels with an “e” as umlaut mark
  \item no overwriting of existing Unicode codepoints
  %\item \href{http://www.e-welt.net/bfds_2003/bund/normsondz/Normblatt 
%UNZ 1 - Antiqua.pdf}{UNZ1} compatibility
 \end{itemize}


 \addsec{Font Faces}
 \addsec*{Frunge Mixtura}

This is a draft and learning example. Its outlines are drawn as if their 
ſtrokes were written with a thick feather, hold by 45 degrees (is to be 
decreased) only, in the lower and upper right direction.

It can be altered through options heavily (not implemented yet).


 \addsec{\texorpdfstring{\XeTeX-Usage}{XeTeX-Usage}}
 The uſage of the fonts with \XeTeX\ is quite eaſy. If you know how to 
uſe \TeX, just load the package 
\href{http://www.ctan.org/tex-archive/macros/xetex/latex/xltxtra/}{\pkg{xltxtra}} 
(which loads the \pkg{fontspec}-package, which does moſt of the work). 
Load the font ſaying \verb|\setmainfont{Frunge Font}|, where 
\verb|Frunge Font| is the name of the font as it is known to your 
operating ſyſtem. This will change the main font of your document to the 
given font. Say \verb|\fontspec{Frunge Font}| to change to the font 
locally.

The uſage of advanced OpenType features will be described in this place 
as ſoon as they are implemented in the fonts.
 %smcp
 %salt


 \addsec{Internal Implementation}
 Accented\,/\,Modified letters are diſſolved via a ccmp-table and 
reſurrected by abvm/blwm/mkmk/mark-tables. Therefore your font rendering 
engine ſhould fully ſupport OpenType (otherwiſe no glyphs are ſhown for 
thoſe). …


 \addsec{Typing of Long s}

%Xmodmap …

Neo—ergonomically optimized keyboard layout for German (and Engliſh); 
ſets long s on firſt level and makes blackletter typing eaſy. 
\href{http://wiki.neo-layout.org/wiki/Neo\%20unter\%20Linux\%20einrichten/Lang-s-Tastatur}{Inſtallation 
inſtructions for GNU/Linux [German]}


 \addsec{Testing of Fonts}
 You can teſt the Frunge fonts in this document with the command 
\verb|\frungepangram| wich will input the file \verb|frunge-pangram|:
 \begin{quote}
  \frungepangram
 \end{quote}
 This (non-ſenſical) text has all important glyphs to ſhow how a 
blackletter font would look like. The text may change from time to time 
as we might find better words \winkey


 \addsec{Similar Projects}
 \begin{itemize}
  \item 
\href{http://fontforge.ſourceforge.net/sfds/toyfonts.html}{FontForge’s 
Toy Fonts}
  \item 
\href{http://www.ctan.org/tex-archive/fonts/gothic/yfrak/}{Yannis 
Haralambous’s Blackletter Metafonts}
  \item \href{http://www.peter-wiegel.de/Fonts.html}{Peter Wiegel’s 
Blackletter and Kurrent fonts}
 \end{itemize}


 \addsec{Contact}
 If you want to contribute ſome patches or feature or bug requeſts, 
viſit our \href{http://ſv.nongnu.org/p/frunge}{project ſite}\\
 \hypertarget{contact}{You can write us email, too: 
\href{mailto:frunge-external@nongnu.org}{frunge-external@nongnu.org}}.\\ 
If you want to contribute\,/\,diſcuſs on a regular baſis, ſubſcribe to 
our 
\href{http://liſts.nongnu.org/mailman/listinfo/frunge-internal}{mailing 
liſt}.

 \hypertarget{fdl}{\addpart{GNU Free Documentation Licenſe}}

                GNU Free Documentation Licenſe
                 Verſion 1.3, 3 November 2008

 Copyright (C) 2000, 2001, 2002, 2007, 2008 Free Software Foundation, Inc.
     <http://fsf.org/>
 Everyone is permitted to copy and diſtribute verbatim copies
 of this licenſe document, but changing it is not allowed.

\addsec{0. PREAMBLE}

The purpoſe of this Licenſe is to make a manual, textbook, or other
functional and uſeful document “free” in the ſenſe of freedom: to
aſſure everyone the effective freedom to copy and rediſtribute it,
with or without modifying it, either commercially or noncommercially.
Secondarily, this Licenſe preſerves for the author and publiſher a way
to get credit for their work, while not being conſidered reſponſible
for modifications made by others.

This Licenſe is a kind of “copyleft”, which means that derivative
works of the document muſt themſelves be free in the ſame ſenſe. It
complements the GNU General Public Licenſe, which is a copyleft
licenſe deſigned for free ſoftware.

We have deſigned this Licenſe in order to uſe it for manuals for free
ſoftware, becauſe free ſoftware needs free documentation: a free
program ſhould come with manuals providing the ſame freedoms that the
ſoftware does. But this Licenſe is not limited to ſoftware manuals;
it can be uſed for any textual work, regardleſs of ſubject matter or
whether it is publiſhed as a printed book. We recommend this Licenſe
principally for works whoſe purpoſe is inſtruction or reference.

\addsec{1. APPLICABILITY AND DEFINITIONS}

This Licenſe applies to any manual or other work, in any medium, that
contains a notice placed by the copyright holder ſaying it can be
diſtributed under the terms of this Licenſe. Such a notice grants a
world-wide, royalty-free licenſe, unlimited in duration, to uſe that
work under the conditions ſtated herein. The “Document”, below,
refers to any ſuch manual or work. Any member of the public is a
licenſee, and is addreſſed as “you”. You accept the licenſe if you
copy, modify or diſtribute the work in a way requiring permiſſion
under copyright law.

A “Modified Version” of the Document means any work containing the
Document or a portion of it, either copied verbatim, or with
modifications and/or tranſlated into another language.

A “Secondary Section” is a named appendix or a front-matter ſection of
the Document that deals excluſively with the relationſhip of the
publiſhers or authors of the Document to the Document’s overall
ſubject (or to related matters) and contains nothing that could fall
directly within that overall ſubject. (Thus, if the Document is in
part a textbook of mathematics, a Secondary Section may not explain
any mathematics.)  The relationſhip could be a matter of hiſtorical
connection with the ſubject or with related matters, or of legal,
commercial, philoſophical, ethical or political poſition regarding
them.

The “Invariant Sections” are certain Secondary Sections whoſe titles
are deſignated, as being thoſe of Invariant Sections, in the notice
that ſays that the Document is releaſed under this Licenſe. If a
ſection does not fit the above definition of Secondary then it is not
allowed to be deſignated as Invariant. The Document may contain zero
Invariant Sections. If the Document does not identify any Invariant
Sections then there are none.

The “Cover Texts” are certain ſhort paſſages of text that are liſted,
as Front-Cover Texts or Back-Cover Texts, in the notice that ſays that
the Document is releaſed under this Licenſe. A Front-Cover Text may
be at moſt 5 words, and a Back-Cover Text may be at moſt 25 words.

A “Tranſparent” copy of the Document means a machine-readable copy,
repreſented in a format whoſe ſpecification is available to the
general public, that is ſuitable for reviſing the document
ſtraightforwardly with generic text editors or (for images compoſed of
pixels) generic paint programs or (for drawings) ſome widely available
drawing editor, and that is ſuitable for input to text formatters or
for automatic tranſlation to a variety of formats ſuitable for input
to text formatters. A copy made in an otherwiſe Tranſsparent file
format whoſe markup, or abſence of markup, has been arranged to thwart
or diſcourage ſubſequent modification by readers is not Tranſparent.
An image format is not Tranſparent if uſed for any ſubſtantial amount
of text. A copy that is not “Tranſparent” is called “Opaque”.

Examples of ſuitable formats for Tranſparent copies include plain
ASCII without markup, Texinfo input format, LaTeX input format, SGML
or XML uſing a publicly available DTD, and ſtandard-conforming ſimple
HTML, PoſtScript or PDF deſigned for human modification. Examples of
tranſparent image formats include PNG, XCF and JPG. Opaque formats
include proprietary formats that can be read and edited only by
proprietary word proceſſors, SGML or XML for which the DTD and/or
proceſſing tools are not generally available, and the
machine-generated HTML, PoſtScript or PDF produced by ſome word
proceſſors for output purpoſes only.

The “Title Page” means, for a printed book, the title page itſelf,
plus ſuch following pages as are needed to hold, legibly, the material
this Licenſe requires to appear in the title page. For works in
formats which do not have any title page as ſuch, “Title Page” means
the text near the moſt prominent appearance of the work’s title,
preceding the beginning of the body of the text.

The “publiſher” means any perſon or entity that diſtributes copies of
the Document to the public.

A section “Entitled XYZ” means a named ſubunit of the Document whoſe
title either is preciſely XYZ or contains XYZ in parentheſes following
text that tranſlates XYZ in another language. (Here XYZ ſtands for a
ſpecific ſection name mentioned below, ſuch as “Acknowledgements”,
"Dedications”, “Endorſements”, or “Hiſtory”.)  To “Preſerve the Title"
of ſuch a ſection when you modify the Document means that it remains a
ſection “Entitled XYZ” according to this definition.

The Document may include Warranty Diſclaimers next to the notice which
ſtates that this Licenſe applies to the Document. Theſe Warranty
Diſclaimers are conſidered to be included by reference in this
Licenſe, but only as regards diſclaiming warranties: any other
implication that theſe Warranty Diſclaimers may have is void and has
no effect on the meaning of this Licenſe.

\addsec{2. VERBATIM COPYING}

You may copy and diſtribute the Document in any medium, either
commercially or noncommercially, provided that this Licenſe, the
copyright notices, and the licenſe notice ſaying this Licenſe applies
to the Document are reproduced in all copies, and that you add no
other conditions whatſoever to thoſe of this Licenſe. You may not uſe
technical meaſures to obſtruct or control the reading or further
copying of the copies you make or diſtribute. However, you may accept
compenſation in exchange for copies. If you diſtribute a large enough
number of copies you muſt alſo follow the conditions in ſection 3.

You may alſo lend copies, under the ſame conditions ſtated above, and
you may publicly diſplay copies.

\addsec{3. COPYING IN QUANTITY}

If you publiſh printed copies (or copies in media that commonly have
printed covers) of the Document, numbering more than 100, and the
Document’s licenſe notice requires Cover Texts, you muſt encloſe the
copies in covers that carry, clearly and legibly, all theſe Cover
Texts: Front-Cover Texts on the front cover, and Back-Cover Texts on
the back cover. Both covers muſt alſo clearly and legibly identify
you as the publiſher of theſe copies. The front cover muſt preſent
the full title with all words of the title equally prominent and
viſible. You may add other material on the covers in addition.
Copying with changes limited to the covers, as long as they preſerve
the title of the Document and ſatisfy theſe conditions, can be treated
as verbatim copying in other reſpects.

If the required texts for either cover are too voluminous to fit
legibly, you ſhould put the firſt ones liſted (as many as fit
reaſonably) on the actual cover, and continue the reſt onto adjacent
pages.

If you publiſh or diſtribute Opaque copies of the Document numbering
more than 100, you muſt either include a machine-readable Tranſparent
copy along with each Opaque copy, or ſtate in or with each Opaque copy
a computer-network location from which the general network-uſing
public has acceſs to download uſing public-ſtandard network protocols
a complete Tranſparent copy of the Document, free of added material.
If you uſe the latter option, you muſt take reaſonably prudent ſteps,
when you begin diſtribution of Opaque copies in quantity, to enſure
that this Tranſparent copy will remain thus acceſſible at the ſtated
location until at leaſt one year after the laſt time you diſtribute an
Opaque copy (directly or through your agents or retailers) of that
edition to the public.

It is requeſted, but not required, that you contact the authors of the
Document well before rediſtributing any large number of copies, to
give them a chance to provide you with an updated verſion of the
Document.

\addsec{4. MODIFICATIONS}

You may copy and diſtribute a Modified Verſion of the Document under
the conditions of ſections 2 and 3 above, provided that you releaſe
the Modified Verſion under preciſely this Licenſe, with the Modified
Verſion filling the role of the Document, thus licenſing diſtribution
and modification of the Modified Verſion to whoever poſſeſſes a copy
of it. In addition, you muſt do theſe things in the Modified Version:

A. Uſe in the Title Page (and on the covers, if any) a title diſtinct
   from that of the Document, and from thoſe of previous verſions
   (which ſhould, if there were any, be liſted in the Hiſtory ſection
   of the Document). You may uſe the ſame title as a previous verſion
   if the original publiſher of that verſion gives permiſſion.
B. Liſt on the Title Page, as authors, one or more perſons or entities
   reſponsible for authorſhip of the modifications in the Modified
   Version, together with at leaſt five of the principal authors of the
   Document (all of its principal authors, if it has fewer than five),
   unleſs they releaſe you from this requirement.
C. State on the Title page the name of the publiſher of the
   Modified Version, as the publiſher.
D. Preſerve all the copyright notices of the Document.
E. Add an appropriate copyright notice for your modifications
   adjacent to the other copyright notices.
F. Include, immediately after the copyright notices, a licenſe notice
   giving the public permiſſion to uſe the Modified Version under the
   terms of this Licenſe, in the form ſhown in the Addendum below.
G. Preſerve in that license notice the full liſts of Invariant Sections
   and required Cover Texts given in the Document’s licenſe notice.
H. Include an unaltered copy of this Licenſe.
I. Preſerve the ſection Entitled “Hiſtory”, Preſerve its Title, and add
   to it an item ſtating at leaſt the title, year, new authors, and
   publiſher of the Modified Verſion as given on the Title Page. If
   there is no ſection Entitled “History” in the Document, create one
   ſtating the title, year, authors, and publiſher of the Document as
   given on its Title Page, then add an item deſcribing the Modified
   Verſion as ſtated in the previous ſentence.
J. Preſerve the network location, if any, given in the Document for
   public acceſs to a Tranſparent copy of the Document, and likewiſe
   the network locations given in the Document for previous verſions
   it was baſed on. Theſe may be placed in the “Hiſtory” ſection.
   You may omit a network location for a work that was publiſhed at
   leaſt four years before the Document itſelf, or if the original
   publiſher of the verſion it refers to gives permiſſion.
K. For any ſection Entitled “Acknowledgements” or “Dedications”,
   preſerve the Title of the ſection, and preſerve in the ſection all
   the ſubſtance and tone of each of the contributor acknowledgements
   and/or dedications given therein.
L. Preſerve all the Invariant Sections of the Document,
   unaltered in their text and in their titles. Section numbers
   or the equivalent are not conſidered part of the ſection titles.
M. Delete any ſection Entitled “Endorſements”. Such a ſection
   may not be included in the Modified Verſion.
N. Do not retitle any exiſting ſection to be Entitled “Endorſements"
   or to conflict in title with any Invariant Section.
O. Preſerve any Warranty Diſclaimers.

If the Modified Verſion includes new front-matter ſections or
appendices that qualify as Secondary Sections and contain no material
copied from the Document, you may at your option deſignate ſome or all
of theſe ſections as invariant. To do this, add their titles to the
liſt of Invariant Sections in the Modified Verſion’s licenſe notice.
Theſe titles muſt be diſtinct from any other ſection titles.

You may add a ſection Entitled “Endorſements”, provided it contains
nothing but endorſements of your Modified Verſion by various
parties—for example, ſtatements of peer review or that the text has
been approved by an organization as the authoritative definition of a
ſtandard.

You may add a paſſage of up to five words as a Front-Cover Text, and a
paſſage of up to 25 words as a Back-Cover Text, to the end of the liſt
of Cover Texts in the Modified Verſion. Only one paſſage of
Front-Cover Text and one of Back-Cover Text may be added by (or
through arrangements made by) any one entity. If the Document already
includes a cover text for the ſame cover, previouſly added by you or
by arrangement made by the ſame entity you are acting on behalf of,
you may not add another; but you may replace the old one, on explicit
permiſſion from the previous publiſher that added the old one.

The author(s) and publiſher(s) of the Document do not by this Licenſe
give permiſſion to uſe their names for publicity for or to aſſert or
imply endorſement of any Modified Verſion.

\addsec{5. COMBINING DOCUMENTS}

You may combine the Document with other documents releaſed under this
Licenſe, under the terms defined in ſection 4 above for modified
verſions, provided that you include in the combination all of the
Invariant Sections of all of the original documents, unmodified, and
liſt them all as Invariant Sections of your combined work in its
licenſe notice, and that you preſerve all their Warranty Diſclaimers.

The combined work need only contain one copy of this Licenſe, and
multiple identical Invariant Sections may be replaced with a ſingle
copy. If there are multiple Invariant Sections with the ſame name but
different contents, make the title of each ſuch ſection unique by
adding at the end of it, in parentheſes, the name of the original
author or publiſher of that ſection if known, or elſe a unique number.
Make the ſame adjuſtment to the ſection titles in the liſt of
Invariant Sections in the licenſe notice of the combined work.

In the combination, you muſt combine any ſections Entitled “Hiſtory"
in the various original documents, forming one ſection Entitled
"Hiſtory"; likewiſe combine any ſections Entitled “Acknowledgements”,
and any ſections Entitled “Dedications”. You muſt delete all ſections
Entitled “Endorſements”.

\addsec{6. COLLECTIONS OF DOCUMENTS}

You may make a collection conſiſting of the Document and other
documents releaſed under this Licenſe, and replace the individual
copies of this Licenſe in the various documents with a ſingle copy
that is included in the collection, provided that you follow the rules
of this Licenſe for verbatim copying of each of the documents in all
other reſpects.

You may extract a ſingle document from ſuch a collection, and
diſtribute it individually under this Licenſe, provided you inſert a
copy of this Licenſe into the extracted document, and follow this
Licenſe in all other reſpects regarding verbatim copying of that
document.

\addsec{7. AGGREGATION WITH INDEPENDENT WORKS}

A compilation of the Document or its derivatives with other ſeparate
and independent documents or works, in or on a volume of a ſtorage or
diſtribution medium, is called an “aggregate” if the copyright
reſulting from the compilation is not uſed to limit the legal rights
of the compilation’s uſers beyond what the individual works permit.
When the Document is included in an aggregate, this Licenſe does not
apply to the other works in the aggregate which are not themſelves
derivative works of the Document.

If the Cover Text requirement of ſection 3 is applicable to theſe
copies of the Document, then if the Document is leſs than one half of
the entire aggregate, the Document’s Cover Texts may be placed on
covers that bracket the Document within the aggregate, or the
electronic equivalent of covers if the Document is in electronic form.
Otherwise they muſt appear on printed covers that bracket the whole
aggregate.


\addsec{8. TRANSLATION}

Tranſlation is conſidered a kind of modification, ſo you may
diſtribute tranſlations of the Document under the terms of ſection 4.
Replacing Invariant Sections with tranſlations requires ſpecial
permiſſion from their copyright holders, but you may include
tranſlations of ſome or all Invariant Sections in addition to the
original verſions of theſe Invariant Sections. You may include a
tranſlation of this Licenſe, and all the licenſe notices in the
Document, and any Warranty Diſclaimers, provided that you alſo include
the original Engliſh verſion of this Licenſe and the original verſions
of thoſe notices and diſclaimers. In caſe of a diſagreement between
the tranſlation and the original verſion of this Licenſe or a notice
or diſclaimer, the original verſion will prevail.

If a ſection in the Document is Entitled “Acknowledgements”,
"Dedications”, or “Hiſtory”, the requirement (ſection 4) to Preſerve
its Title (ſection 1) will typically require changing the actual
title.


\addsec{9. TERMINATION}

You may not copy, modify, ſublicenſe, or diſtribute the Document
except as expreſſly provided under this Licenſe. Any attempt
otherwiſe to copy, modify, ſublicenſe, or diſtribute it is void, and
will automatically terminate your rights under this Licenſe.

However, if you ceaſe all violation of this Licenſe, then your licenſe
from a particular copyright holder is reinſtated (a) proviſionally,
unleſs and until the copyright holder explicitly and finally
terminates your licenſe, and (b) permanently, if the copyright holder
fails to notify you of the violation by ſome reaſonable means prior to
60 days after the ceſſation.

Moreover, your licenſe from a particular copyright holder is
reinſtated permanently if the copyright holder notifies you of the
violation by ſome reaſonable means, this is the firſt time you have
received notice of violation of this Licenſe (for any work) from that
copyright holder, and you cure the violation prior to 30 days after
your receipt of the notice.

Termination of your rights under this ſection does not terminate the
licenſes of parties who have received copies or rights from you under
this Licenſe. If your rights have been terminated and not permanently
reinſtated, receipt of a copy of ſome or all of the ſame material does
not give you any rights to uſe it.


\addsec{10. FUTURE REVISIONS OF THIS LICENSE}

The Free Software Foundation may publiſh new, reviſed verſions of the
GNU Free Documentation Licenſe from time to time. Such new verſions
will be ſimilar in ſpirit to the preſent verſion, but may differ in
detail to addreſs new problems or concerns. See
http://www.gnu.org/copyleft/.

Each verſion of the Licenſe is given a diſtinguiſhing verſion number.
If the Document ſpecifies that a particular numbered verſion of this
Licenſe “or any later verſion” applies to it, you have the option of
following the terms and conditions either of that ſpecified verſion or
of any later verſion that has been publiſhed (not as a draft) by the
Free Software Foundation. If the Document does not ſpecify a verſion
number of this Licenſe, you may chooſe any verſion ever publiſhed (not
as a draft) by the Free Software Foundation. If the Document
ſpecifies that a proxy can decide which future verſions of this
Licenſe can be uſed, that proxy’s public ſtatement of acceptance of a
verſion permanently authorizes you to chooſe that verſion for the
Document.

\addsec{11. RELICENSING}

"Maſſive Multiauthor Collaboration Site” (or “MMC Site") means any
World Wide Web ſerver that publiſhes copyrightable works and alſo
provides prominent facilities for anybody to edit thoſe works. A
public wiki that anybody can edit is an example of ſuch a ſerver. A
"Maſſive Multiauthor Collaboration” (or “MMC") contained in the ſite
means any ſet of copyrightable works thus publiſhed on the MMC ſite.

"CC-BY-SA” means the Creative Commons Attribution-Share Alike 3.0 
licenſe publiſhed by Creative Commons Corporation, a not-for-profit 
corporation with a principal place of buſineſs in San Franciſco, 
California, as well as future copyleft verſions of that licenſe 
publiſhed by that ſame organization.

"Incorporate” means to publiſh or republiſh a Document, in whole or in 
part, as part of another Document.

An MMC is “eligible for relicenſing” if it is licenſed under this 
Licenſe, and if all works that were firſt publiſhed under this Licenſe 
ſomewhere other than this MMC, and ſubſequently incorporated in whole or 
in part into the MMC, (1) had no cover texts or invariant ſections, and 
(2) were thus incorporated prior to November 1, 2008.

The operator of an MMC Site may republiſh an MMC contained in the ſite
under CC-BY-SA on the ſame ſite at any time before Auguſt 1, 2009,
provided the MMC is eligible for relicenſing.


\addsec{ADDENDUM: How to uſe this Licenſe for your documents}

To uſe this Licenſe in a document you have written, include a copy of
the Licenſe in the document and put the following copyright and
licenſe notices juſt after the title page:

    Copyright (c)  YEAR  YOUR NAME.
    Permission is granted to copy, distribute and/or modify this document
    under the terms of the GNU Free Documentation Licenſe, Version 1.3
    or any later version published by the Free Software Foundation;
    with no Invariant Sections, no Front-Cover Texts, and no Back-Cover Texts.
    A copy of the license is included in the section entitled “GNU
    Free Documentation License”.

If you have Invariant Sections, Front-Cover Texts and Back-Cover Texts,
replace the “with … Texts.” line with this:

    with the Invariant Sections being LIST THEIR TITLES, with the
    Front-Cover Texts being LIST, and with the Back-Cover Texts being LIST.

If you have Invariant Sections without Cover Texts, or ſome other
combination of the three, merge thoſe two alternatives to ſuit the
ſituation.

If your document contains nontrivial examples of program code, we
recommend releaſing theſe examples in parallel under your choice of
free ſoftware licenſe, ſuch as the GNU General Public Licenſe,
to permit their uſe in free ſoftware.


 \end{document}
